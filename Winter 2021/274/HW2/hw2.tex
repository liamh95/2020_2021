\documentclass[11pt,letterpaper]{report}
\usepackage{amssymb,amsfonts,color,graphicx,amsmath,enumerate}
\usepackage{amsthm}

\newcommand{\naturals}{\mathbb{N}}
\newcommand{\integers}{\mathbb{Z}}
\newcommand{\complex}{\mathbb{C}}
\newcommand{\reals}{\mathbb{R}}
\newcommand{\mcal}[1]{\mathcal{#1}}
\newcommand{\rationals}{\mathbb{Q}}
\newcommand{\field}{\mathbb{F}}
\newcommand{\Var}{\text{Var}}
\newcommand{\ind}{\mathbbm{1}}
\newcommand{\Cov}{\text{Cov}}
\newcommand{\rank}{\text{rank}}

\newenvironment{solution}
{\begin{proof}[Solution]}
{\end{proof}}

\voffset=-3cm
\hoffset=-2.25cm
\textheight=24cm
\textwidth=17.25cm
\addtolength{\jot}{8pt}
\linespread{1.3}

\begin{document}
\noindent{\em Liam Hardiman\hfill{February 15, 2021} }
\begin{center}
{\bf \Large Math 274 - Homework 2}
\vspace{0.2cm}
\hrule
\end{center}

\noindent\textbf{Problem 1. }
Let $v_1 = (x_1, y_1), \ldots, v_n = (x_n, y_n)$ be $n$ vectors in $\integers^2$, where each $x_i$ and each $y_i$ is a positive integer that does not exceed $\frac{2^{n/2}}{10\sqrt{n}}$.
Show that there exist two disjoint nonempty subsets $I, J\subseteq [n]$ such that $\sum_{i\in I}v_i = \sum_{j\in J}v_j$.

\begin{proof}
    Consider the random sum $X = \sum_{i=1}^n \epsilon_iv_i$ where each $\epsilon_i$ is an iid Bernoulli random variable with success probability $1/2$.
    We'll show that a sizeable proportion of the possible $X_i$ live in an axis-aligned box centered about the mean of $X$.
    If the size of this box is smaller than the number of assignments of the $\epsilon_i$'s that make $X$ land in this box, then there must be two assignments of the $\epsilon_i$'s that give the same realization of $X$.\\

    Let 
\end{proof}

\end{document}
