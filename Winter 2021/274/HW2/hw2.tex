\documentclass[11pt,letterpaper]{report}
\usepackage{amssymb,amsfonts,color,graphicx,amsmath,enumerate}
\usepackage{amsthm}

\newcommand{\naturals}{\mathbb{N}}
\newcommand{\integers}{\mathbb{Z}}
\newcommand{\complex}{\mathbb{C}}
\newcommand{\reals}{\mathbb{R}}
\newcommand{\mcal}[1]{\mathcal{#1}}
\newcommand{\rationals}{\mathbb{Q}}
\newcommand{\field}{\mathbb{F}}
\newcommand{\Var}{\text{Var}}
\newcommand{\ind}{\mathbbm{1}}
\newcommand{\Cov}{\text{Cov}}
\newcommand{\rank}{\text{rank}}
\newcommand{\E}{\mathbb{E}}

\newenvironment{solution}
{\begin{proof}[Solution]}
{\end{proof}}

\voffset=-3cm
\hoffset=-2.25cm
\textheight=24cm
\textwidth=17.25cm
\addtolength{\jot}{8pt}
\linespread{1.3}

\begin{document}
\noindent{\em Liam Hardiman\hfill{February 15, 2021} }
\begin{center}
{\bf \Large Math 274 - Homework 2}
\vspace{0.2cm}
\hrule
\end{center}

\noindent\textbf{Problem 1. }
Let $v_1 = (x_1, y_1), \ldots, v_n = (x_n, y_n)$ be $n$ vectors in $\integers^2$, where each $x_i$ and each $y_i$ is a positive integer that does not exceed $\frac{2^{n/2}}{10\sqrt{n}}$.
Show that there exist two disjoint nonempty subsets $I, J\subseteq [n]$ such that $\sum_{i\in I}v_i = \sum_{j\in J}v_j$.

\begin{proof}
    Consider the random sum $V = (X, Y) = \sum_{i=1}^n \epsilon_iv_i$ where each $\epsilon_i$ is an iid Bernoulli random variable with success probability $1/2$.
    We'll show that a sizeable proportion of the possible $V$ live in an axis-aligned box centered about the mean of $V$.
    If the size of this box is smaller than the number of assignments of the $\epsilon_i$'s that make $V$ land in this box, then there must be two assignments of the $\epsilon_i$'s that give the same realization of $V$.\\

    Since $X$ is a sum of independent scaled Bernoulli random variables, its variance is easily computed:
    \[
        \Var[X] = \sum_{i=1}^nx_i^2\cdot \Var[\epsilon_i] \leq \frac{2^n}{400}.
    \]
    Note that the $y$-coordinate has the same variance.
    Now by Chebyshev we have
    \[\Pr\left[ \left|X - \frac{2^n}{400}\right| \leq t\right] \geq 1-\frac{\Var[X]}{t^2},
    \]
    and the same inequality holds for $Y$.
    If we choose $t$ so that both of these probabilities are at least $3/4$, then $V$ will leave in an axis-aligned square centered at $\E[V]$ with width $2t$ with probability at least $1/2$.
    Choosing $t = 2^{n/2}/10$ does precisely this.
    The resulting box contains $\sim 2^n/25$ integer points as well as at least $2^{n-1}$ possible realizations of $V$.
    Since there are more realizations than integer points in this box, two realizations must land on the same integer point.
    In particular, there are some \textit{nonempty} and \textit{distinct} $I', J' \subseteq [n]$ such that $\sum_{i\in I'}v_i = \sum_{j\in J'}v_j$. Setting $I = I'\setminus J'$ and $J = J'\setminus I'$ gives us the desired \textit{disjoint} sets.
\end{proof}










\noindent\textbf{Problem 2. }
Let $G = (V, E)$ be a graph on $n$ vertices and let $0<\alpha<1$.
Consider the following game on the edge set of $G$.
There are two players, Alice and Bob, who alternate turns (starting with Alice) occupying perviously unoccupied edges in $E$ until there are no more edges to claim.
Let $G_A = (V, E_A)$ be the spanning subgraph of $G$ where $E_A$ is the set of edges occupied by Alice at the game's conclusion.
Alice wins if and only if for every vertex $v\in V$, its degree in $G_A$ is at least an $\alpha$-fraction of its original degree in $G$.\\

Show that for large enough $n$, if $G$ has minimum degree at least $\log^2n$, then Alice has a winning strategy for $\alpha\geq 1/3 - \epsilon$.

\begin{proof}
    Consider the following randomized strategy for Alice.
    She opens by taking any random edge in $E$.
    On each of her subsequent turns, she examines the edge $\{x,y\}$ that Bob just claimed.
    She flips a fair coin -- claiming a random unclaimed edge incident to $x$ if it comes up heads and incident to $y$ if tails.
    If neither $x$ nor $y$ has any unclaimed incident edges, Alice simply picks up a random unclaimed edge in $E$.\\

    If Alice wins with this strategy with positive probability, then any realization of winning coin flips corresponds to a \emph{deterministically} winning strategy.
    Now for any vertex $v$ in $V$, let $d_A(v)$ denote its degree in $G_A$ and $d(v)$ its degree in $G$.
    The expected value of $d_A(v)$ is at least $\frac{1}{3}d(v)$ since, on average, Alice claims an edge incident to $v$ at least half as often as Bob does.
    By Chernoff, for any $\epsilon<1$ and any vertex $v$ we have
    \[\Pr[d_1(v) < (1-\epsilon)d(v)/3] \leq e^{-\epsilon^2d(v)/6} \leq n^{-\frac{\epsilon^2}{6}\log n}.\]
    A union bound then gives
    \[
        \Pr[d_1(v)<(1-\epsilon)d(v)/3\text{ for some }v] \leq n^{1-\frac{\epsilon^2}{6}\log n}.
    \]
    Alice loses if she claims less than an $\alpha$-fraction of the neighbors of any vertex $v$ and solving the equation $1-epsilon = 3\alpha$ for $\alpha$ lets us conclude that she wins with positive probability when $\alpha = 1/3 - \epsilon$.
\end{proof}










\noindent\textbf{Problem 3. }
Prove that for every constant $d$, there exists a constant $C$ so that for any given graph $G$ with $|V(G)| = n$ (where $n$ is arbitrarily large) and $\Delta(G)\leq d$, there exists an injective function $f: V\to [Cn]$ such that the quantities $\{|f(u)-f(v)|: uv\in E(G)\}$ are all distinct.

\begin{proof}
    Choose $f$ uniformly at random among all injections $V\to [Cn]$ and for any pair of edges $uv$ and $xy$ in $E$, define $A_{uv,xy}$ to be the event that $|f(u)-f(v)| = |f(x)-f(y)|$.
    Furthermore, let $N(uv, xy)$ denote the set of (unordered) pairs of edges, one of which is equal to either $uv$ or $xy$.
    We claim that for any pair of edges $uv$, $xy$ and any $J\subseteq \binom{E}{2}\setminus N(uv, xy)$, we have
    \[
        \Pr\left[ A_{uv, xy} \;\middle|\; \bigcap_{(u'v', x'y')\in J}A_{u'v', x'y'}^c  \right] \leq \Pr[A_{uv, xy}].
    \]
    Indeed, set $E_J$ to be the event being conditioned on in the above expression for ease of notation.
    We have
    \[
        \Pr[A_{uv,xy} \mid E_J] = \sum_{j=1}^{n-1}\Pr\bigg[|f(u)-f(v)| = |f(x)-f(y)| = j \mid E_J\bigg].
    \]
    Informally, since $f$ is an injection, each $j = 1, \ldots, n-1$ can be realized as a distance $|f(u)-f(v)|$ only so many times (depending on $j$).
    Each $A_{u'v', x'y'}^c$ hidden behind $E_J$ represents a pair of edges, disjoint from $uv, xy$ attaining two different distances under $f$.
    Since $u'v'$ and $x'y'$ attain different distances under $f$, it is more likely that at least one of them is $j$, and consequently, it is less likely that $uv$ and $xy$ are both separated by a distance of $j$ under $f$.
    Thus, the above quantity is at most $\sum_{j=1}^{n-1}\Pr[|f(u)-f(v)| = |f(x)-f(y)| = j] = \Pr[A_{uv,xy}]$.\\

    We have then established that the $A_{uv,xy}$ and $N(uv, xy)$ form a negative dependency graph.
    Now $|N(uv, xy)|$ is the number of edge pairs, one of which is equal to $uv$ or $xy$.
    There are at most $2(m-1) \leq n\Delta-2$ such pairs.
    Let's bound $\Pr[A_{uv, xy}]$ from above.
    If $uv$ and $xy$ have a vertex in common, they are less likely to satisfy $|f(u)-f(v)| = |f(x)-f(y)|$ than if they had no vertex in common.
    We then suppose that $uv$ and $xy$ share no vertex.
    We have
    \begin{align*}
        \Pr[A_{uv, xy}] &= \sum_{j=1}^{n-1}\Pr\bigg[|f(u)-f(v)| = |f(x)-f(y)| = j\bigg]\\
        &\leq \sum_{j=1}^{n-1}\frac{4\cdot \frac{(n-j)^2}{2}\cdot \binom{cn-4}{n-4}(n-4)!}{\binom{cn}{n}n!}\\
        &\sim \frac{2}{n^4}\cdot \frac{\binom{cn-4}{n-4}}{\binom{cn}{n}}\cdot \sum_{j=1}^{n-1}(n-j)^2\\
        &\sim \frac{2}{3nc^4}.
    \end{align*}
    Call this (asymptotic) upper bound $p$.
    Since $ep(n\Delta - 1) \leq \frac{2e\Delta}{3c^4}$, we may simply choose $c$ large enough so that this quantity is less than 1 and apply the Lov\'asz local lemma to conclude that the distances $|f(u)-f(v)|$ are all distinct with positive probability (and large enough $n$).
\end{proof}










\noindent\textbf{Problem 4. }
Prove that for $n$ sufficiently large, any set $A$ of $n$ distinct integers contains two disjoint subsets $B_1$ and $B_2$ satisfying $|B_1| = |B_2| > 0.33n$, where each of the sets $B_i$ is sum-free.

\begin{proof}
    Let $q$ be a prime much larger than the largest element of $A$, say congruent to 1 modulo 6, so $q = 6k+1$ for some $k$.
    Notice that the interval $S_1 = \{2k+1, 2k+2, \ldots, 4k\}$ is sum-free modulo $q$, as is the union of intervals $S_2 = \{k+1, \ldots, 2k-1\}\cup \{4k+1, \ldots, 5k\}$.
    Consider the transformation $T:A\to \integers_q$ where $a\mapsto ax+y \pmod{q}$ where $x$ and $y$ are nonzero integers modulo $q$ chosen independently at random.
    Notice that for any $u,v\in \integers_q$, and any distinct $a_1, a_2$ in $A$, the events $\{T(a_1) = u\}$ and $\{T(a_2) = v\}$ are independent.
    Indeed, these events each occur with probability $1/q$ and
    \[
        \Pr[a_1x+y \equiv u,\ a_2x+y\equiv v] = \Pr\left[\begin{pmatrix}
            a_1 & 1\\
            a_2 & 1
        \end{pmatrix}\begin{pmatrix}
            x\\y
        \end{pmatrix}\equiv \begin{pmatrix}
            u\\v
        \end{pmatrix}\right] = \frac{1}{q^2},
    \]
    since the above matrix has rank 2 (so there is a \textit{unique} pair $x,y$ solving the corresponding system).\\

    Now $B\subseteq A$ is sum-free if $T[B] \subseteq (S_1+y)$.
    Indeed, if $b_1+b_2 = b_3$ for $b_1, b_2, b_3$ in $B$, then $b_ix = s_i$ for some $s_i$ in $S_1$ for $i = 1, 2, 3$ and $s_1+s_2= s_3$, contradicting the fact that $S_1$ is sum-free.
    The same holds for $S_2$.
    Now if we can show that the expected number of $a_i$ such that $T(a_i) \in (S_1+y)$ is at least $n/3$ (and that the same holds for $S_2$), the independence of the events $\{T(a_i) \in (S_1+y)\}$ and $\{T(a_j) \in (S_2+y)\}$ for distinct $a_i$ and $a_j$ will guarantee the existence of disjoint sum-free subsets $B_1$ and $B_2$, both of size around $1/3$.\\

    For any interval $I$ of length $\epsilon n$ in $\integers_q$, the expected size of $T[A]\cap I$ is $\epsilon n^2/q$.
    If we write $|T[A]\cap I| = \sum_{a\in A}X_{a, I}$, where $X_{a, I}$ indicates whether $T(a)\in I$, then we have
    \[
        \Var\bigg[|T[A]\cap I|\bigg]= \sum_{a\in A}\Var[X_{a, I}] = n\cdot \frac{\epsilon n^2}{q} \cdot \left(1 - \frac{\epsilon n^2}{q}\right),
    \]
    since the $X_{a, I}$ are independent by our construction of $T$ from \textit{two} random sources $x$ and $y$.
    By Chebyshev's inequality, we have
    \[
        \Pr\bigg[\big||T[A]\cap I| - \epsilon^2n/q| \geq \epsilon^2 n/q] = o(1),
    \]
    so for $n$ sufficiently large, a $1-o(1)$ fraction of the intervals of $\integers_q$ contain the expected number of elements of $T[A]$.
    Consequently, $|T[A]\cap (S_1+y)|$ and $|T[A]\cap (S_2+y)|$ are both simultaneously near their expected sizes, say equal to $0.33 n$.
\end{proof}










\noindent\textbf{Problem 5. }
Prove that there exists an absolute constant $c>0$ such that for any tournament $T = (V, E)$ on $n$ vertices, there are two disjoint subsets $A,B\subseteq V$ such that $e(A,B)-e(B,A) \geq cn^{3/2}$.

\begin{proof}
    I think the idea here is to choose $A$ and $B$ in some random way and then show that $\E[(e(A,B)-e(B,A))^2] \geq cn^3$.
    There would then be some realization of $A$ and $B$ for which $(e(A,B)-e(B,A))^2$ is at least $cn^3$ and we'd be done.
    I wasn't able to come up with a way to get a variance with order higher than $n^2$, unfortunately.\\

    Intuitively, I wanted two random sets of roughly equal size, so for each element in $V$, we include it in $A$ with probability $p$, in $B$ also with probability $p$, and neither with probability $1-2p$.
    Since $e(A,B)$ and $e(B,A)$ have the same distribution, we have
    \[
        \E[(e(A,B) - e(B,A))^2] = 2\E[e(A,B)^2] - 2\E[e(A,B)e(B,A)].
    \]
    For each edge $e$ we let $X_e$ be the variable that indicates whether $e$ crosses from $A$ to $B$.
    We similarly  let $\tilde{X}_e$ indicate whether $e$ crosses from $B$ to $A$.
    We have
    \[
        \E[e(A,B)^2] = \E\left[\left(\sum_{e\in E}X_e\right)^2\right] = \binom{n}{2}p^2 + \sum_{e\neq f}\E[X_eX_f].
    \]
    If the edges $e$ and $f$ share no common vertex, then $\E[X_eX_f] = p^4$.
    If both are in- or out-edges to a common vertex, then $\E[X_eX_f] = p^3$.
    If one is an in-edge and one is an out-edge, then $\E[X_eX_f] = 0$.
    Now we compute
    \[
        \E[e(A,B)e(B,A)] = \E\left[\left(\sum_{e\in E}X_e\right)\left(\sum_{f\in E}\tilde{X}_f\right)\right] = \sum_{e,f\in E}\E[X_e\tilde{X_f}].
    \]
    If $e = f$ then we clearly have $\E[X_e\tilde{X}_f] = 0$.
    If they don't share a vertex, then we get $p^4$.
    When they're both in- or out-edges of some vertex then we get zero, and if one is an in-edge and the other an out-edge, we get $p^3$.
    Let $S_+$ be the set of all edge pairs that are out-edges to some common vertex, $S_-$ to be the pairs that form in-edges, and $S_{\pm}$ the pairs where one is an in-edge and the other an out-edge.
    Since the contributions due to edge pairs without common vertices cancel with each other in our expressions for $\E[e(A,B)^2]$ and $\E[e(A,B)e(B,A)]$, we have
    \[
        \E[(e(A,B) - e(B,A))^2] = p^2\binom{n}{2}+2p^3\bigg(|S_-| + |S_+| - |S_\pm| \bigg).
    \]
    We can exactly calculate the expression in parentheses above by summing over each vertex ($d^+(v)$ and $d^-(v)$ are the out- and in- degrees of $v$, respectively)
    \begin{align*}
        |S_|+|S_+|-|S_\pm|&= \sum_{v\in V}\left[\binom{d^-(v)}{2} + \binom{d^+(v)}{2} - d^-(v)d^+(v)\right]\\
        &= \frac{1}{2}\sum_{v\in V}(d^+(v)-d^-(v))^2 - \frac{1}{2}\sum_{v\in V}(d^-(v)+d^+(v))\\
        &= \frac{1}{2}\sum_{v\in V}(d^+(v)-d^-(v))^2 - \binom{n}{2}.
    \end{align*}

    Our second moment is then
    \[
        \E[(e(A,B) - e(B,A))^2] = \binom{n}{2}(p^2-2p^3) + p^3\sum_{v\in V}(d^+(v)-d^-(v))^2.
    \]
    By considering the case of a regular tournament (wherein $d^+(v) = d^-(v)$ for all $v\in V$), we see that the above quantity can't be made larger than $\Theta(n^2)$ in general.
    Interestingly, if we set $p = 1/2$ and consider a regular tournament, we obtain a variance of zero, which concludes a circuitous proof for the fact that $e(A,A^c)-e(A^c,A) = 0$ for any subset $A$ in this case.\\

    I must have the wrong experiment for choosing $A$ and $B$.
    This regular tournament example tells me that we can't do something as simple as take $B = A^c$.
    I think setting $p = 1/3$ above chooses disjoint subsets $A$ and $B$ of $V$ uniformly at random and this still doesn't work.
\end{proof}


\end{document}
