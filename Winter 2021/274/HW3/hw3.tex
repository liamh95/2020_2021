\documentclass[11pt,letterpaper]{report}
\usepackage{amssymb,amsfonts,color,graphicx,amsmath,enumerate}
\usepackage{amsthm}

\newcommand{\naturals}{\mathbb{N}}
\newcommand{\integers}{\mathbb{Z}}
\newcommand{\complex}{\mathbb{C}}
\newcommand{\reals}{\mathbb{R}}
\newcommand{\mcal}[1]{\mathcal{#1}}
\newcommand{\rationals}{\mathbb{Q}}
\newcommand{\field}{\mathbb{F}}
\newcommand{\Var}{\text{Var}}
\newcommand{\ind}{\mathbbm{1}}
\newcommand{\Cov}{\text{Cov}}
\newcommand{\rank}{\text{rank}}
\newcommand{\E}{\mathbb{E}}

\theoremstyle{plain}
\newtheorem{theorem}{Theorem}[section]
\newtheorem{lemma}[theorem]{Lemma}

\newenvironment{solution}
{\begin{proof}[Solution]}
{\end{proof}}

\voffset=-3cm
\hoffset=-2.25cm
\textheight=24cm
\textwidth=17.25cm
\addtolength{\jot}{8pt}
\linespread{1.3}

\begin{document}
\noindent{\em Liam Hardiman\hfill{March 20, 2021} }
\begin{center}
{\bf \Large Math 274 - Homework 3}
\vspace{0.2cm}
\hrule
\end{center}

\noindent\textbf{Problem 1. }
Prove that for any positive integer $k>1$ there is a $c = c(k)$ so that for any collection of subsets $A_1, \ldots, A_k\subseteq \{0, 1\}^n$ that that satisfy $|A_i| \geq 2^n/k$ for all $i$, there are points $v_i\in A_i$ such that any pair of points $v_i, v_j$, $i\neq j$, differ in at most $c\sqrt{n}$ coordinates.



\begin{proof}
    Recall the following corollary to Azuma's inequality.
    \begin{theorem}\label{thm: Ham}
        Suppose $\epsilon$ and $\lambda$ are positive real numbers satisfying $e^{-\lambda^2/2} = \epsilon$.
        Then for any $A\subseteq \{0,1\}^n$ of size at least $\epsilon\cdot 2^n$, we have
        \[
            |B(A, 2\lambda \sqrt{n})| \geq (1-\epsilon)2^n,
        \]
    \end{theorem}
    where $B(S, r)$ is the set of all strings in $\{0,1\}^n$ with Hamming distance at most $r$ from some string in $S$.
    From this we deduce that $|B(A_j, 2\lambda \sqrt{n})| \geq (1-1/k)2^n$ for all $j$, where $\lambda = \sqrt{2\log k}$.\\

    We start by showing that $S_1 = \cap_{j\geq 2} B(A_j, 2\lambda \sqrt{n})$ is nonempty (and is, in fact, quite large).
    From the above discussion and a simple union bound we deduce the following.
    \[
        \left|\bigcup_{j\geq 2}B(A_j, 2\lambda \sqrt{n})^c\right| \leq \sum_{j\geq 2}|B(A_j, 2\lambda \sqrt{n})^c| \leq (1-1/k)2^n.
    \]
    Hence, $|S_1| \geq 2^n/k$ and we can again apply Theorem \ref{thm: Ham} to obtain $|B(S_1, 2\lambda \sqrt{n})| \geq (1-1/k)2^n$.
    Since $|A_1| = 2^n/k$, if $A_1$ and $B(S_1, 2\lambda \sqrt{n})$ do not intersect, then $A_1^C = B(S_1, 2\lambda\sqrt{n})$ and then $A_1$ and $B(S_1, 2\lambda\sqrt{n}+1)$ intersect.\\

    By our definition of $S_1$, there is then some $v_1$ in $A_1$ and some $v$ in $\cap_{j\geq 2}B(A_j, 2\lambda \sqrt{n})$ so that $d(v_1, v) \leq 4\lambda \sqrt{n}$.
    Finally, we may choose $v_j$ in $A_j$ for $j\geq 2$ so that $d(v_j, v)\leq 2\lambda\sqrt{n}$ and
    \[
        d(v_i, v_j) \leq d(v_i, v) + d(v_j, v) \leq 6\lambda \sqrt{n}
    \]
    for all $i\neq j$.
\end{proof}






\noindent\textbf{Problem 2. }
Prove that if $M$ is an $n\times n$ matrix over some finite field $\field$ with $per(M)\neq 0$, then for every vector $b\in \field^n$ there exists $x\in \{0, 1\}^n$ for which every coordinate $i$ in $Mx$ is distinct from $b_i$.


\begin{proof}
    For each $b\in \field^n$ consider the polynomial $f_b: \field^n\to \field$ given by
    \[
        f_b(x) = \prod_{i=1}^n((Mx)_i - b_i) = \prod_{i=1}^n\left(\sum_{j=1}^nM_{ij}x_j - b_i\right).
    \]
    The degree of $f_b$ is $n$ and the coefficient of the term $x_1\cdots x_n$ is $per(M)$.
    To see this, note that we obtain this coefficient by summing over all possible ways to pick $A_{ij}x_j$ exactly once from each of the $n$ factors in the product that defines $f_b$.
    The desired coefficient is then
    \[
        \sum_{\sigma \in S_n}\prod_{j=1}^nM_{j\sigma(j)} = perm(M).
    \]
    Since the coefficient of $x_1^1\cdots x_n^1$ is nonzero and $|\{0,1\}| = 2$, there is an $x\in \{0,1\}^n$ such that $f_b(x) \neq 0$, which corresponds to a $0/1$ vector for which $M_x$ differs from $b$ in every coordinate.
\end{proof}







\noindent\textbf{Problem 3. }
Let $H = (V, E)$ be a hypergraph where each edge is of size $t$ and each vertex has degree at most $t$.
Show that
\[
    disc(H) = O(\sqrt{t\log t}).
\]

\begin{proof}
    
\end{proof}








\noindent\textbf{Problem 4. }
Fix $n\in \naturals$.
We say that $P(n)$ is true if for any $a_1, \ldots, a_{2n-1}\in \integers$, there is an $I\subseteq [2n-1]$ with $\sum_{i\in I}a_i \equiv 0 \pmod{n}$ and $|I| = n$.
Show that if $P(n)$ and $P(m)$ are true, then so is $P(nm)$.










\noindent\textbf{Problem 5. }
A 1-factorization in a hypergraph $H = (V,E)$ is a collection of edge-disjoint perfect matchings that cover all the edges of $H$.
Let $K_n^k$ denote the complete $k$-uniform hypergraph on $n$ vertices.
Our goal is to prove the following theorem.
\begin{theorem}
    Let $k$ and $n$ be two positive integers for which $n$ is divisible by $k$.
    Then the complete $k$-uniform hypergraph on $n$ vertices admits a 1-factorization.
\end{theorem}

\begin{enumerate}[(a)]
    \item Prove the following lemma.
    \begin{lemma}
        For any real $m\times n$ matrix $M$ with integer row and column sums, there is an integer $m\times n$ matrix $M'$ having the same row and column sums as $M$ and satisfying
        \[
            |m_{ij}-m'_{ij}|<1,\quad \forall i,j.
        \]
    \end{lemma}
\end{enumerate}


\end{document}
