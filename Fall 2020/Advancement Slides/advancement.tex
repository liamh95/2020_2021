\documentclass{beamer}
\usepackage{amssymb,amsfonts,color,graphicx,amsmath,enumerate,mathtools}
\usepackage{tikz} %This package offers the ability to draw pictures
\usepackage{amsthm}
\usepackage{hyperref}
\usepackage{lmodern}


\theoremstyle{plain}
\newtheorem{claim}[theorem]{Claim}
\newtheorem{proposition}[theorem]{Proposition}
\newtheorem{observation}[theorem]{Observation}
\newtheorem{conjecture}[theorem]{Conjecture}
\newtheorem{remark}[theorem]{Remark}
\newtheorem{property}[theorem]{Property}
\newtheorem{exercise}[theorem]{Exercise}
\newtheorem{exercises}[theorem]{Exercises}
\newcommand{\Bin}{\ensuremath{\textrm{Bin}}}
\newcommand{\Bern}{\text{Bern}}


\newcommand{\naturals}{\mathbb{N}}
\newcommand{\Z}{\mathbb{Z}}
\newcommand{\complex}{\mathbb{C}}
\newcommand{\reals}{\mathbb{R}}
\newcommand{\exreals}{\overline{\mathbb{R}}}
\newcommand{\mcal}[1]{\mathcal{#1}}
\newcommand{\mable}{measurable}
\newcommand{\quats}{\mathbb{H}}
\newcommand{\rationals}{\mathbb{Q}}
\newcommand{\norm}{\trianglelefteq}
\newcommand{\Aut}{\text{Aut}}
\newcommand{\disk}{\mathbb{D}}
\newcommand{\halfplane}{\mathbb{H}}
\newcommand{\Lp}[2]{\left\|{#1}\right\|_{L^{#2}}}
\newcommand{\supp}[1]{\text{supp}({#1})}
\newcommand{\Hom}[2]{\text{Hom}_{{#1}}({#2})}
\newcommand{\tr}{\text{tr}}
\newcommand{\field}[1]{\mathbb{F}_{{#1}}}
\newcommand{\Gal}[1]{\text{Gal}\left({#1}\right)}
\newcommand{\esssup}{\text{ess sup }}
\newcommand{\essinf}{\text{ess inf }}
\newcommand{\affine}{\mathbb{A}}
\newcommand{\E}{\mathbb{E}}
\newcommand{\Var}{\text{Var}}


\title{Finding and Counting Substructures in Graphs and Hypergraphs}
\author{Liam Hardiman}
\date{December 10, 2020}

\usetheme{Frankfurt}

\AtBeginSection[]{
	\begin{frame}<beamer>
		\tableofcontents[currentsection]
	\end{frame}
}

\begin{document}

\maketitle

% Roadmap
% Definitions
% Random Graphs Paper
%   Background - Dirac, Gallai, Scott
%   Big induced graph
%   Packing
% Hamiltonian Cycles
%   Dirac's theorem in graphs
%   How many ham cycles in graphs
%   How2cycle in a hypergraph
%   counting in hypergraphs


\section{A Finding Problem}
	\begin{frame}{Quick Definitions}
		\begin{itemize}
			\item A \textbf{graph} $G = (V, E)$ consists of a (finite) set of \textbf{vertices} $V$ and a set $E$ of unordered pairs of vertices called \textbf{edges}.\pause

			\item If $X\subseteq V$, then the \textbf{subgraph induced by $X$}, $G[X]$ is the graph with vertex set $X$ and all edges from $E$ that have both ends in $X$.\pause

			\item The \textbf{degree} of a vertex $v$, denoted $d_G(v)$, is the number of edges of $G$ that $v$ appears in.\pause

			\item picture here
		\end{itemize}
	\end{frame}


	\begin{frame}{History and Motivation}
		\begin{theorem}[L. Lov\'asz, T. Gallai - 1979]
			Let $G = (V, E)$ be any graph.
			Then $G$ admits a partitioning of its vertex set into two parts, $V = V_1 \cup V_2$, so that each vertex in $G[V_1]$ and each vertex in $G[V_2]$ has even degree.
			In particular, any graph on $n$ vertices has an even subgraph of order at least $n/2$.
		\end{theorem}\pause

		\begin{proof}[Proof sketch:]
			asdf
		\end{proof}
	\end{frame}


	\begin{frame}{History and Motivation}
		\begin{itemize}
			\item What about odd subgraphs?\pause

			\begin{theorem}[A. Scott - 1992]
				Every graph $G(V, E)$ on $n$ vertices, none of which are isolated, contains a set $W\subseteq V(G)$ such that $|W|\geq \frac{n}{900 \log n}$ and $G[W]$ has all degrees odd.
			\end{theorem}\pause

			\begin{conjecture}[A. Scott - 2001]
				There exists some constant $c>0$ such that every graph $G(V, E)$ on $n$ vertices, none of which are isolated, contains a set $W\subseteq V(G)$ such that $|W|\geq cn$ and $G[W]$ has all degrees odd.
			\end{conjecture}\pause
		\end{itemize}
	\end{frame}


	\begin{frame}{History and Motivation}
		\begin{definition}
			The \textbf{Erd\H{o}s-Renyi random graph}, $\mathcal{G}(n, p)$, is the random variable that outputs a graph on $n$ vertices, any two of which are independently connected with probability $p$.
		\end{definition}\pause

		\begin{theorem}[A. Scott - 1992]
			Let $G\sim \mathcal{G}(n, 1/2)$.
			Then with high probability (that is, with probability $1-o(1)$), $G$ has an induced subgraph on at least $0.7729n$ vertices with all degrees odd.
		\end{theorem}
	\end{frame}


	\begin{frame}{Other Moduli and Remainders}
		\begin{conjecture}[A. Scott - 2001]
			For any positive integer $q$ at least 2, there exists a constant $c_q$ so that every graph on $n$ vertices without isolated vertices has an induced subgraph on at least $c_qn$ vertices with all degrees $1\pmod q$.\pause
		\end{conjecture}
		% \begin{definition}
		% 	Let $q$ be a positive integer at least 2 and let $0\leq r<q$ be an integer.
		% 	If $G$ is a graph, we write $f(G, r, q)$ for the maximum order of an induced subgraph of $G$ with all degrees $r\pmod q$.
		% \end{definition}
		\begin{theorem}[A. Ferber, H., M. Krivelevich - 2020+]
			Let $q\geq 2$ and let $r$ be an integer.
			Then there exists a constant $c_q$ such that, with high probability, the random graph $G\sim \mathcal{G}(n, 1/2)$ has an induced subgraph on at least $c_qn$ vertices wijth all degrees $r\pmod q$.
		\end{theorem}
	\end{frame}


	\begin{frame}{Key Idea Behind Proof}
		\begin{itemize}
			\item If $G\sim \mathcal{G}(n, 1/2)$, then its adjacency matrix $M$ ($M_{ij} = 1$ if and only vertices $i$ and $j$ are connected and $M_{ij}=0$ otherwise) is a random symmetric $n\times n$ matrix whose above-diagonal entries are iid $\Bern(1/2)$ random variables.

			\pause

			\item In this case, the $i$-th entry of $M\boldsymbol{1}$ is the degree of vertex $i$ in $G$, where $\boldsymbol{1}$ is the length $n$ all-1 vector.

			\pause

			\begin{observation}
				$G$ contains a subgraph with all degrees $r\pmod q$ if and only if its adjacency matrix contains a principal submatrix $B$ satisfying $B\boldsymbol{1} \equiv r\boldsymbol{1}\pmod q$.
			\end{observation}
		\end{itemize}
	\end{frame}


	\begin{frame}{A Useful Computation}
		Fix a positive integer $q\geq 2$ and let $r$ be an integer.
		Let $x_1, \ldots, x_t$ be iid $\Bern(1/2)$ random variables.

		\pause

		\begin{align*}
			\Pr[x_1 + \cdots + x_t \equiv r] &= \E[\delta_0(x_1 + \cdots + x_t - r)]\\
			\uncover<3->{&= \frac{1}{q}\sum_{\ell\in \Z_q}\E\exp\left[\frac{2\pi i\ell}{q}(x_1 + \cdots  + x_t - r) \right]\\}
			\uncover<4->{&=\frac{1}{q}\sum_{\ell\in \Z_q}e^{-2\pi ir \ell/q}\prod_{j=1}^t\E e^{2\pi i \ell x_j/q}\\}
			\uncover<5->{&= \frac{1}{q}\sum_{\ell\in \Z_q}e^{-2\pi ir \ell/q}\left(\frac{1+e^{2\pi i\ell/q}}{2}\right)^t.}
		\end{align*}
	\end{frame}


	\begin{frame}{A Useful Computation}
		\begin{itemize}
			\item Isolate the $\ell \equiv 0$ term, apply the triangle inequality and Euler's identity.

			\pause

			\begin{align*}
				\left|\Pr[x_1 + \cdots + x_t \equiv r] - \frac{1}{q}\right|\leq \frac{1}{q}\sum_{\ell = 1}^{q-1}|\cos(\pi \ell/q)|^t.
			\end{align*}

			\pause

			\item Calculus trick: when $|x|\leq \pi/2$, $|\sin x| \geq (2/\pi)|x|$.

			\pause

			\item From this, it follows that $|\cos(\pi \ell/q)|\leq e^{-2/q^2}$ for all $\ell = 1, \ldots, q-1$.

			\pause

			\[
				\left|\Pr[x_1 + \cdots + x_t \equiv r] - \frac{1}{q}\right|\leq \frac{q-1}{q}e^{-2t/q^2}.
			\]
		\end{itemize}
	\end{frame}


	\begin{frame}{A Useful Computation}
		\begin{itemize}
			\item The distribution of a random Bernoulli sum modulo $q$ is asymptotically uniform!

			\pause

			\item By independence we have.

			\begin{lemma}
				Let $M$ be an $s\times t$ matrix whose entries are iid $\Bern(1/2)$ random variables.
				Then for any $v\in \Z_q^s$,
				\[
					\Pr[M\boldsymbol{1} \equiv v] = \frac{1}{q^s}\bigg(1 + O\left(e^{-2t/q^2}\right)\bigg)^s.
				\]
			\end{lemma}	
		\end{itemize}
	\end{frame}


	\begin{frame}{Symmetric Case}
		\begin{itemize}
			\item But the adjacency matrix is symmetric!

			\pause

			\item A similar, but more technical argument handles the symmetric case.

			\pause

			\begin{lemma}
				Let $M$ be an $m\times m$ symmetric matrix whose diagonal is zero and whose entries above the diagonal are iid $\Bern(1/2)$ random variables.
				Then for any $v\in \Z_q^m$,
				\[
					\Pr[M\boldsymbol{1} \equiv v] = \begin{cases}
						\frac{1}{q^m}(1 + O(e^{-m/q^2})) & \text{ if $q$ is odd}\\
						\frac{2}{q^m}(1 + O(e^{-m/q^2})) & \text{ if $q$ is even and $\sum v_i$ is even}\\
						0 & \text{ if $q$ is even and $\sum v_i$ is odd}.
					\end{cases}
				\]
			\end{lemma}
		\end{itemize}
	\end{frame}


	\begin{frame}{Parity?}
		\begin{itemize}
			\item We have
			\[
				\sum_i (M\boldsymbol{1})_i = \sum_{i,j} M_{ij} = 2\sum_{i<j}M_{ij},
			\]
			which is always even modulo $q$ if $q$ is even and can be any residue modulo odd $q$.

			\pause

			\item The symmetric lemma then says that the distribution of $M\boldsymbol{1}\pmod q$ is asymptotically uniform over all ``feasible'' values.
		\end{itemize}
	\end{frame}


	\begin{frame}{Proof (Sketch) of Theorem}
		\begin{itemize}
			\begin{theorem}[Chebyshev's Inequality]
				Let $X$ be a nonnegative integer-valued random variable with finite variance.
				Then
				\[
					\Pr[X > 0] \geq 1 - \frac{\Var[X]}{(\E[X])^2}.
				\]

			\end{theorem}

			\pause

			\item Let $k = c_qn$ for some $c_q>0$ and let $X_k$ be the number of $k\times k$ principal submatrices $B$ of the adjacency matrix of $G$ satisfying $B\boldsymbol{1} \equiv r\boldsymbol{1} \pmod {q}$.

			\pause

			\item Show that $c_q$ can be chosen so that $\Var[X_k] = o(E[X_k]^2)$.
			Then $G$ has an induced subgraph of size $c_qn$ with high probability.
		\end{itemize}
	\end{frame}


	\begin{frame}{Proof (Sketch) of Theorem}
		Maybe have some of the calculations and that overlap diagram.	
	\end{frame}


	\begin{frame}
		State theorem for arbitrary distribution of residues mod $q$.	
	\end{frame}


\section{A Counting Problem}

	\begin{frame}{Packing}
		\begin{itemize}
			\item Recall Gallai's result on even subgraphs.
			\begin{theorem}[L. Lov\'asz, T. Gallai - 1979]
				Let $G = (V, E)$ be any graph.
				Then $G$ admits a partitioning of its vertex set into two parts, $V = V_1 \cup V_2$, so that each vertex in $G[V_1]$ and each vertex in $G[V_2]$ has even degree.
				In particular, any graph on $n$ vertices has an even subgraph of order at least $n/2$.
			\end{theorem}\pause

			\item What else can we say about partitions and remainders?
		\end{itemize}
	\end{frame}


	\begin{frame}{Packing Theorems}
		\begin{theorem}[A. Scott - 2001]
			Let $G$ be a graph. Then $G$ admits a partition into odd subgraphs if and only if every component of $G$ has even order.
		\end{theorem}

		\pause

		\begin{theorem}[A. Scott - 2001]
			With high probability, $G\sim \mathcal{G}(n, p)$ for $n$ even admits a partition into three odd subgraphs.
		\end{theorem}

		\pause

		\begin{conjecture}[A. Scott - 2001]
			For $q>1$ and $r \leq q$, there exists a constant $c_q$ such that $G\sim \mathcal{G}(n, p)$ admits a partition into $c_q$ classes such that the degrees in each class are congruent to $r\pmod q$ with high probability.
		\end{conjecture}
	\end{frame}


	\begin{frame}{Packing Theorems}
		\begin{theorem}[A. Ferber, H., M. Krivelevich]
			For $q>1$ and $r\leq q$, $G\sim \mathcal{G}(n, 1/2)$ admits a partition into $q+1$ classes such that the degrees in each class are congruent to $r\pmod q$ with high probability.
		\end{theorem}
	\end{frame}


	\begin{frame}{Sketch of Packing Proof}
		Explain that it's just another second moment argument. Maybe some calculations, diagrams. Explain need for ``both ways'' lemma.
	\end{frame}


\section{Another Counting Problem}

	\begin{frame}{Definitions}
		\begin{definition}
			Let $G = (V, E)$ be a graph.
			A subset of edges $\mathcal{M}\subseteq E$ is called a \textbf{matching} if its edges are vertex-disjoint.
			The matching $\mathcal{M}$ is \textbf{perfect} if the vertices that comprise it cover all of $V$.
		\end{definition}

		\pause

		\begin{definition}
			Let $G = (V, E)$ be a graph and let $\mathcal{C}$ be a sequence of edges.
			Then $\mathcal{C}$ is a...
			\begin{itemize}
				\item ...\textbf{path} if it joins a sequence of distinct vertices.
				\pause
				\item ...\textbf{cycle} if it joints a sequence of vertices $v_1, v_2, \ldots, v_k, v_1$.
				\pause
				\item ...\textbf{Hamiltonian cycle} if it is a cycle that visits every vertex in $G$ exactly once.
			\end{itemize}
		\end{definition}
	\end{frame}


	\begin{frame}{History and Motivation}
		\begin{itemize}
			\item When does a graph have Hamiltonian a cycle (when is it \textbf{Hamiltonian})?

			\pause

			\item For any graph $G$, let $\delta(G)$ denote the minimum degree of any vertex.

			\pause

			\begin{theorem}[G. Dirac - 1952]
				A graph on $n\geq 3$ vertices is Hamiltonian if $\delta(G)\geq n/2$.
			\end{theorem}

			\pause

			\item A graph is Hamiltonian if it is sufficiently dense.
		\end{itemize}
	\end{frame}


	\begin{frame}{Counting Hamiltonian Cycles in Dense Graphs - Intuition}
		\begin{itemize}
			\item The expected number of Hamiltonian cycles in $G\sim \mathcal{G}(n, p)$ is
			\[
			\frac{1}{2}(n-1)!p^n.
			\]

			\pause

			\item Fix $\gamma>0$.
			If $p>(1/2+\gamma)$, then 
		\end{itemize}
	\end{frame}

	\begin{frame}{Cuckler-Kahn}
		
	\end{frame}


	\begin{frame}{Introduce hypergraphs and notion of cycle}
		
	\end{frame}


	\begin{frame}{Hypergraph theorems}
		
	\end{frame}


	\begin{frame}{Our theorem}
		
	\end{frame}


	\begin{frame}{Proof sketch}
		
	\end{frame}


\end{document}
