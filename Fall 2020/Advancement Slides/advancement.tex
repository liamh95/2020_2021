\documentclass{beamer}
\usepackage{amssymb,amsfonts,color,graphicx,amsmath,enumerate,mathtools}
\usepackage{tikz} %This package offers the ability to draw pictures
\usepackage{amsthm}
\usepackage{hyperref}


\theoremstyle{plain}
\newtheorem{claim}[theorem]{Claim}
\newtheorem{proposition}[theorem]{Proposition}
\newtheorem{observation}[theorem]{Observation}
\newtheorem{conjecture}[theorem]{Conjecture}
\newtheorem{remark}[theorem]{Remark}
\newtheorem{property}[theorem]{Property}
\newtheorem{exercise}[theorem]{Exercise}
\newtheorem{exercises}[theorem]{Exercises}
\newcommand{\Bin}{\ensuremath{\textrm{Bin}}}
\newcommand{\Bern}{\text{Bern}}


\newcommand{\naturals}{\mathbb{N}}
\newcommand{\integers}{\mathbb{Z}}
\newcommand{\complex}{\mathbb{C}}
\newcommand{\reals}{\mathbb{R}}
\newcommand{\exreals}{\overline{\mathbb{R}}}
\newcommand{\mcal}[1]{\mathcal{#1}}
\newcommand{\mable}{measurable}
\newcommand{\quats}{\mathbb{H}}
\newcommand{\rationals}{\mathbb{Q}}
\newcommand{\norm}{\trianglelefteq}
\newcommand{\Aut}{\text{Aut}}
\newcommand{\disk}{\mathbb{D}}
\newcommand{\halfplane}{\mathbb{H}}
\newcommand{\Lp}[2]{\left\|{#1}\right\|_{L^{#2}}}
\newcommand{\supp}[1]{\text{supp}({#1})}
\newcommand{\Hom}[2]{\text{Hom}_{{#1}}({#2})}
\newcommand{\tr}{\text{tr}}
\newcommand{\field}[1]{\mathbb{F}_{{#1}}}
\newcommand{\Gal}[1]{\text{Gal}\left({#1}\right)}
\newcommand{\esssup}{\text{ess sup }}
\newcommand{\essinf}{\text{ess inf }}
\newcommand{\affine}{\mathbb{A}}


\title{Finding and Counting Substructures in Graphs and Hypergraphs}
\author{Liam Hardiman}
\date{December 10, 2020}

\usetheme{Frankfurt}

\AtBeginSection[]{
	\begin{frame}<beamer>
		\tableofcontents[currentsection]
	\end{frame}
}

\begin{document}

\maketitle

% Roadmap
% Definitions
% Random Graphs Paper
%   Background - Dirac, Gallai, Scott
%   Big induced graph
%   Packing
% Hamiltonian Cycles
%   Dirac's theorem in graphs
%   How many ham cycles in graphs
%   How2cycle in a hypergraph
%   counting in hypergraphs


\section{A Finding Problem}
	\begin{frame}{Quick Definitions}
		\begin{itemize}
			\item A \textbf{graph} $G = (V, E)$ consists of a (finite) set of \textbf{vertices} $V$ and a set $E$ of unordered pairs of vertices called \textbf{edges}.\pause

			\item If $X\subseteq V$, then the \textbf{subgraph induced by $X$}, $G[X]$ is the graph with vertex set $X$ and all edges from $E$ that have both ends in $X$.\pause

			\item The \textbf{degree} of a vertex $v$, denoted $d_G(v)$, is the number of edges of $G$ that $v$ appears in.\pause

			\item picture here
		\end{itemize}
	\end{frame}


	\begin{frame}{History and Motivation}
		\begin{theorem}[L. Lov\'asz, T. Gallai - 1979]
			Let $G = (V, E)$ be any graph.
			Then $G$ admits a partitioning of its vertex set into two parts, $V = V_1 \cup V_2$, so that each vertex in $G[V_1]$ and each vertex in $G[V_2]$ has even degree.
			In particular, any graph on $n$ vertices has an even subgraph of order at least $n/2$.
		\end{theorem}\pause

		\begin{proof}[Proof sketch:]
			asdf
		\end{proof}
	\end{frame}


	\begin{frame}{History and Motivation}
		\begin{itemize}
			\item What about odd subgraphs?\pause

			\begin{theorem}[A. Scott - 1992]
				Every graph $G(V, E)$ on $n$ vertices, none of which are isolated, contains a set $W\subseteq V(G)$ such that $|W|\geq \frac{n}{900 \log n}$ and $G[W]$ has all degrees odd.
			\end{theorem}\pause

			\begin{conjecture}[A. Scott - 2001]
				There exists some constant $c>0$ such that every graph $G(V, E)$ on $n$ vertices, none of which are isolated, contains a set $W\subseteq V(G)$ such that $|W|\geq cn$ and $G[W]$ has all degrees odd.
			\end{conjecture}

		\end{itemize}
	\end{frame}


\end{document}
